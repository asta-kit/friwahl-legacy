\documentclass[a4paper,10pt]{article}
\usepackage{german}
\usepackage{a4wide}
\usepackage{wasysym}
\usepackage[utf8]{inputenc}

\parindent0cm
\thispagestyle{empty}

\begin{document}

\section*{WLAN-Troubleshooting}
\textit{Du bist hier richtig, falls du die Wahl-CD mit einer WLAN-Verbindung betreiben willst, aber nach dem Bootvorgang keine Serververbindung aufgebaut werden kann. Bei anderen Problemen, oder wenn die Anweisungen in diesem Dokument nicht helfen, bitte die Hotline anrufen.}

\begin{itemize}
	\item Alle Einstellungen zur Problemlösung sind auf \textit{Konsole 2} vorzunehmen (erreichbar durch Drücken von \texttt{[Alt]+[F2]})
	\item Eine Funktionierende Verbindung ist in der Statusanzeige auf \textit{Konsole 2} an \textit{VPNC: läuft Server: erreichbar} zu erkennen. \textit{Server: erreichbar} ohne ein laufendes VPNC genügt nicht!
	\item Nach einem erfolgreichen Verbindungsaufbau ist auf \textit{Konsole 1} (\texttt{[Alt]+[F1]}) die Fehlermeldung sowie das Neustarten des Programms zu bestätigen.
\end{itemize}

\subsection*{Problemübersicht}

\begin{itemize}
 \item \texttt{Die Statusanzeige zeigt unter \textit{IP:} entweder \textit{???/0.0.0.0} oder \textit{lo/127.0.0.1}:}

\begin{tabular}{lp{11.8cm}}
	Problem: & WLAN nicht verbunden \\
	Lösung: &
	Zunächst auf \textit{Konsole 3} wechseln (\texttt{[Alt]+[F3]}). Wird auf dieser Konsole \textit{fatal error: no wireless interfaces found!} angezeigt, wird der WLAN-Chipsatz des verwendeten Laptops nicht unterstützt ($\Rightarrow$ einen anderen Laptop verwenden!) Falls das nicht der Fall ist, bitte lesen was unter \textit{ESSID:} steht. Steht hier ein Eintrag der Form \textit{dukath-??}, wurde ein DuKaTH in der Umgebung gefunden. Wenn nicht, ist entweder kein WLAN in der Umgebung verfügbar ($\Rightarrow$ Kabel legen!) oder der WLAN-Treiber unterstützt das Scannen nach verfügbaren Netzwerken nicht ($\Rightarrow$ einen anderen Laptop verwenden!). Wurde ein Netzwerk in der Umgebung gefunden, ist weiterhin auf \textit{Konsole 3} zu schauen, ob unter \textit{ip:} etwas anderes steht als \textit{0.0.0.0}. Falls ja, wurde die Statusanzeige nicht korrekt ausgelesen ($\Rightarrow$ ein wenig warten und auf \textit{Konsole 2} wechseln und die Statusanzeige betrachten. Steht dort \textit{0.0.0.0}, hat die automatische Vergabe von \textit{IP-Adressen} nicht funktioniert. Dies kann Vorkommen, wenn der Empfang schlecht ist, oder das Netzwerk überlastet ist. Dazu sind die drei Balken auf \textit{Konsole 3} zu beachten (grün=gut, rot=schlecht). Ein erneuter Verbindungsversuch ist zu starten über \textit{Konsole 2} unter \textit{Experten $\rightarrow$ NetworkManager neustarten}. Etwas warten, bis sich der Status einige male akualisiert hat. Es kann bis zu 1min dauern, bis eine Verbindung vorhanden ist. Wenn über diesen Weg eine Verbindung hergestellt werden kann, muss noch eine VPN hergestellt werden ($\Rightarrow$ siehe nächster Abschnitt)
\end{tabular}

\item \texttt{Die Statusanzeige zeigt \textit{VPNC: läuft nicht}:}

\begin{tabular}{lp{11.8cm}}
	Problem: & VPN nicht verbunden \\
	Lösung: &
	\texttt{VPN Probleme} auswählen und bestätigen. Kommt die Meldung \textit{vpn.uni-karlsruhe.de nicht erreichbar} oder Vergleichbares, besteht ein Problem mit der WLAN-Verbindung ($\Rightarrow$ \textit{Experten $\rightarrow$ NetworkManager neustarten}, warten bis Verbindung wieder aufgebaut ist des öfteren, anschließend erneut \textit{VPN Probleme} auswählen)
\end{tabular}

\end{itemize}

\end{document}
