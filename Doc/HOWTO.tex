
\documentclass[a4paper,10pt]{article}
\usepackage{german}
\usepackage{a4wide}
\usepackage{wasysym}
\usepackage[utf8]{inputenc}

\title{Wahlsoftware - HOWTO \\ \small{\textit{oder: Wieso heißt der Wahlserver eigentlich Client?}}}
\author{Mario Prausa (Wahl 2009)}

\begin{document}

\maketitle
\tableofcontents

\section*{Vorwort}
Dieses HOWTO soll dazu dienen, dem Wahlausschuss den Einstieg in die Arbeit mit der Wahlsoftware zu erleichtern. Zur Wahl 2009 standen dem Wahlausschuss unzählige unstrukturierte Textdokumente zur Verfügung, auf denen dieses HOWTO basiert (vielen Dank an die Autorenschaft). Diese waren aber zum Teil veraltet oder unvollständig. Da in dieser Wahl die Wahl-CD von \textit{Knoppix} auf \textit{lArch} umgestellt wurde, war es notwendig eine neue Dokumentation zu erstellen. Es ist ausdrücklich von jedem zukünftigem Wahlausschuss erwünscht, die Informationen in diesem Dokument zu aktualisieren, zu erweitern und zu vervollständigen (du darfst dich dann auch oben als Autor eintragen).
\newpage

\section{Aufsetzen des Wahlservers}

\subsection{Debian Installation}
Als erstes musst du auf dem Wahlserver ein aktuelles \textit{Debian} installieren. Da das aktuelle \textit{etch} einen zu alten Kernel (Stand Dezember 2008) für den Paketmanager Pacman hat, der für die Erzeugung der Wahl-CDs benötigt wird, ist die Installation von \textit{etch'n'half} zu empfehlen. Zur Sicherheit sollte am besten ein \textit{RAID-1} verwendet werden. \\
Während der Installation kann dann direkt das Netzwerk konfiguriert werden. Dafür wird \textit{DHCP} verwendet. Der DHCP-Server sollte die \textit{MAC-Adresse} des Servers bereits kennen, und ist somit anschließend ohne weiteres UStA-intern über \texttt{wahl.usta.de} und extern (d.h. innerhalb des Uni-Netzes) über \texttt{asta-wahl.asta.uni-karlsruhe.de} erreichbar. \\
Nach der Installation muss mindestens ein Benutzer für den Wahlausschuss angelegt werden (siehe \texttt{man adduser}), den wir der Einfachheit halber \textit{admin} nennen. Falls mehrere Mitglieder des Wahlausschusses Zugang zum Wahlserver benötigen, sollte jeder von diesen einen eigenen Benutzer bekommen. \\
Nun musst du die für die Wahlsoftware wichtigen Pakete installieren (\texttt{apt-get install <paket>}):
\begin{description}
	\item[Benötigte Pakete] dialog, postgresql, ssh, subversion, rsync, mkisofs, libcdk-perl, libdate-manip-perl, libpg-perl, openssl, ntpdate, imagemagick, gsfonts, sudo
	\item[Sicherheitskram] debsums, aide, tiger, chkrootkit, logcheck
 \end{description}
und natürlich dein Lieblingseditor!
\vspace{8pt} \\
Nun müssen einige weitere Konfigurationen vorgenommen werden:
\begin{itemize}
	\item \texttt{/etc/mailname} auf \texttt{wahl.usta.de} setzen
	\item die erste Zeile in \texttt{/etc/hosts} sollte nur noch
		\texttt{127.0.0.1 localhost} lauten
	\item In der Datei \texttt{/etc/wgetrc} die Option \texttt{passive\_ftp} auf
		\texttt{off} setzen
	\item Mittels 
	\begin{verbatim}
		curl-config --ftp-port
	\end{verbatim} das Defaultverhalten von \texttt{--ftp-pass} auf \texttt{--ftp-port} setzten.
	\item \texttt{aide -c /etc/aide/aide.conf --init} aufrufen. \\
		Anschließend die Datei \texttt{/var/lib/aide/aide.db.new} nach
		\texttt{/var/lib/aide/aide.db} kopieren
\end{itemize}

\subsection{Installation der Wahlsoftware}
Checke als Benutzer \textit{root} im Home-Verzeichnis die aktuelle Wahlsoftware aus:
\begin{verbatim}
	svn co svn+ssh://<benutzer>@login.usta.de/data/svn/wahl
\end{verbatim} 
Die aktuelle Wahlsoftware befindet sich nun entweder unterhalb des Verzeichnisses \texttt{\textasciitilde/wahl/trunk/} oder in einem extra Branch unterhalb von \texttt{\textasciitilde/wahl/branches/}.
im Verzeichnis \texttt{Client/} befindet sich die Server-Software (alles klar?). Die Skripte zur Erzeugung der Wahl-CD befinden sich im Verzeichnis \texttt{FriWahlCD/} und im Verzeichnis \texttt{AuszaehlSys/} befindet sich, wie der Name schon sagt, das Auszähl System. Der SymLink \texttt{UrneFrontend/} führt direkt zur Client Software unterhalb von \texttt{FriWahlCD/}. \\
Nun installieren wir die Server-Software, indem wir im Verzeichnis \texttt{Client/} das Skript \texttt{setup.sh} ausführen. Auf die Frage ``Welche Accounts gehören zum Wahlausschuss?'', sind die Benutzer, die weiter oben angelegt wurden, anzugeben (z.B. \textit{admin}, aber nicht \textit{root}). Die Wahlsoftware ist nun unter \texttt{/usr/local/FriCardWahl/} installiert.

\newpage
\section{Erstellen der Wahl-CDs}

\subsection{Aufbau}
Im Verzeichnis \texttt{FriWahlCD/} befinden sich die Skripte zur Erzeugung der Wahl-CDs. Die Wahl-CDs basieren auf \textit{lArch} (\texttt{larch.berlios.de}), eine Skriptsammlung zum Erstellen von Live-CDs auf Basis von \textit{ArchLinux} (\texttt{www.archlinux.org}). Damit diese Skripte wissen, was auf die CD drauf soll, gibt es die Möglichkeit Profile anzulegen. Das Profil für die Wahl-CD befindet sich unter \texttt{usta/profile/}. Die zwei wichtigsten Bestandteile eines solchen Profils sind: Die Datei \texttt{addedpacks}, die eine Liste von Paketen enthält, die auf dem Live-System installiert werden sollen, sowie das Verzeichnis \texttt{rootoverlay/}. Wie auf Live-CDs üblich, wird das Dateisystem des Systems mit Hilfe von \textit{SquashFS} (\texttt{http://de.wikipedia.org/wiki/SquashFS}) gepackt, bevor es auf der CD gespeichert wird. \textit{lArch} verfolgt die Philosophie, das System zu squashen, bevor es konfiguriert wurde. Da \textit{SquashFS} nicht beschreibbar ist (und eine CD schon recht nicht), wird dieses Dateisystem beim Booten nun via \textit{aufs} (\texttt{http://de.wikipedia.org/wiki/Aufs}) mit einer RAM-Disk überlagert. Nach dem nun ein (simulierter) Schreibzugriff möglich ist, wird das sogenannte \textit{Rootoverlay} entpackt, und ``überschreibt'' damit die darin doppelt vorhandenen Dateien auf dem \textit{SquashFS}. Über diesen Mechanismus kann nun das Live-System konfiguriert werden, indem darin die benötigten Konfigurationsdateien angelegt werden. In dem \textit{Rootoverlay}, dessen Dateien sich im Verzeichnis \texttt{usta/profile/rootoverlay/} befinden, ist auch das Urnen Frontend unter \texttt{usr/local/usta/} installiert. Da jede Urne eine eigene Wahl-CD mit eigenen Keys im \textit{Rootoverlay} benötigt, allerdings die zu installierenden Pakete immer die selben sind, ist das Erstellen der CDs in zwei Schritte gegliedert. Der erste Schritt beinhaltet das Installieren der Pakete, was nur einmal durchgeführt werden muss. Der zweite Schritt erzeugt das \textit{Rootoverlay} und muss individuell für jede Urne ausgeführt werden. Aber zuerst müssen ein paar Konfigurationen vorgenommen werden.

\subsection{Konfiguration}
In der Datei \texttt{people.dat} findet sich eine Liste von Namen, die während dem Bootvorgang mit Hilfe von \textit{Splashy} angezeigt werden. Die Datei besteht aus einem Eintrag pro Zeile, wobei jede Zeile in 3 Felder aufgeteilt ist, die durch einen Doppelpunkt getrennt sind. Im ersten Feld steht entweder \texttt{A} oder \texttt{L} für Wahl\textbf{A}usschuss bzw. Wahl\textbf{L}eiter. Im 2. Feld steht der Name der Person und im 3. Feld deren Funktion bzw. die Fachschaft, für die sie Wahlleiter ist. Da \textit{Splashy} bereits gestartet wird, bevor das \textit{Rootoverlay} geladen ist, muss das angezeigte Bild, das aus den Namen gerendert wird, bereits im \textit{SquashFS} installiert werden. Deshalb gilt die Option \texttt{-s} von \texttt{makecd.sh} zu beachten, falls Änderungen im Nachhinein an dieser Datei vorgenommen werden. \\
Desweiteren wird die Datei \texttt{usta/data/server} dann nützlich sein, wenn zu Testzwecken ein anderer Server wie der offizielle Wahlserver verwendet werden soll (Einfach Server-Adresse eintragen... fertig).

\subsection{Vorbereiten der CDs}
Zuerst wird einmal die \textit{lArch}-Skriptsammlung benötigt. Um an diese zu gelangen, einfach in das Verzeichnis \texttt{larch/} wechseln und \texttt{./larch-setup} ausführen (Achtung: nicht aus einem anderen Verzeichnis heraus starten, da die Skripte unterhalb des aktuellen Verzeichnisses installiert werden). Nun reicht es aus im Verzeichnis \texttt{FriWahlCD/} das Skript \texttt{./preparecd.sh} auszuführen. Nun gehst du zum Kühlschrank und holst dir ein Bier. Nachdem du es ausgetrunken hast, sind vielleicht die ca. 200 Pakete installiert. Aber zum Glück muss man das in der Regel nur einmal für alle Urnen machen. Installiert wird dieses System unter \texttt{workdir/build/}. Das \textit{SquashFS} wird in diesem Schritt noch nicht generiert.

\subsection{Erzeugen der Urnen-ISOs}
Nun kommen wir zum Erstellen der individuellen Urnen-CDs. Dafür gibt es das Skript \texttt{makecd.sh}. Ein Aufruf von \texttt{./makecd.sh} ohne Parameter liefert:
\begin{verbatim}
	Usage: ./makecd.sh urne [konf_account] [konf_pw] [-s]
\end{verbatim}
Der erste Parameter \texttt{urne} ist der Benutzername mit dem sich die Wahl-CD beim Server anmeldet. Er ist von der Form \texttt{urneXX}. Die \texttt{XX} stehen dabei für die Urnennummer (führende $0$ bei Urne $1$-$9$ nicht vergessen!). \texttt{konf\_account} und \texttt{konf\_pw} sind die Zugangsdaten für einen RZ-Account. Zu Testzwecken kann hier auch der persönliche RZ-Account genommen werden. Für die Wahl ist für jede Urne ein eigener Konferenzaccount erforderlich (Bitte dafür an den UStA-Admin wenden). Beim Ausführen von \texttt{makecd.sh} wird im Home-Verzeichnis für die Urne ein Unterverzeichnis in \texttt{\textasciitilde/keys/} und in \texttt{\textasciitilde/accounts} mit dem Benutzernamen der Urne erstellt, in denen die für den Login auf dem Server benötigten Keys sowie die Accountdaten für das RZ abgelegt werden. Ein erneuter Aufruf von \texttt{makecd.sh} (z.B. nach Änderungen am \textit{Rootoverlay}) verwendet diese Dateien wieder und ignoriert die auf der Kommandozeile angegeben RZ-Zugangsdaten. Deshalb können bei späteren Aufrufen diese Parameter weggelassen werden. \texttt{makecd.sh} kopiert das Profil nach \texttt{workdir/profile.urneXX/} und installiert darin dann die Keys im \textit{Rootoverlay}.
Beim ersten Aufruf bei der ersten Urne wird aus dem Verzeichnis \texttt{workdir/build/} das \textit{SquashFS} erzeugt. Für die weiteren Urnen wird dieses wiederverwendet, es sei denn die Option \texttt{-s} wird verwendet (z.B. bei Änderungen in der \texttt{people.dat} notwendig).
Die fertige ISO ist anschließend unter \texttt{WAHL-CD.urneXX.iso} zu finden. \\
Für Testzwecken ist empfohlen die Urne $99$ zu verwenden. Auf die Angabe des Urnenusers kann dabei verzichtet werden, wenn anstatt auf \texttt{makecd.sh} auf das Skript \texttt{makecd-demo.sh} zurückgegriffen wird. \\
Das Skript \texttt{cds} ist ein Wrapper für \texttt{makecd.sh} der verwendet wird, um gleich einige CDs auf einmal zu erzeugen. Dafür wird eine Datei benötigt die pro Zeile einen RZ-Account und das zugehörige Passwort durch ein Leerzeichen getrennt enthält. Das Skript erwartet als ersten Parameter den Pfad zu der Account-Datei, der zweite Parameter ist ein Pfad zu einem entfernten Rechner (z.B. \texttt{gast@aphrodite:ein\_pfad/}) auf den mit scp die fertigen ISOs geladen werden (dieser sollte natürlich einen Brenner haben). Als dritter und vierter Parameter kann die Start- und Endnummer der zu erstellenden Urnen angegeben werden.

\newpage
\section{Wahlen erstellen}
\subsection{Wahlen vorbereiten}
\begin{LARGE}TODO\end{LARGE}
\subsection{Urnen registrieren}
\begin{LARGE}TODO\end{LARGE}

\newpage
\section{Auszählsystem}
\begin{LARGE}TODO\end{LARGE}

\newpage
\section{Anpassen des Systems}
\subsection{Arch Build System}
Benötigst du für die Wahl-CD ein Paket, das nicht in den Arch Repositories zu finden ist, kommst du nicht darum herum dieses selbst zu bauen. Dafür bietet \textit{Arch Linux} das sogenannte \textit{Arch Build System} (\textit{ABS}) an. Um dieses zu nutzen, ist eine \textit{Arch Linux} Umgebung notwendig. Wenn du kein Rechner mit \textit{Arch Linux} zur Verfügung hast, gibt es die Möglichkeit eine Arch Umgebung in einem \textit{chroot} mit Hilfe des Skripts \texttt{FriWahlCD/usta/scripts/archchroot} einzurichten. Der erste Parameter ist die Zielarchitektur und muss für die Wahl-CD auf \texttt{i686} gesetzt werden. Der zweite Parameter ist das Zielverzeichnis. Im Zielverzeichnis findet sich nach dem Durchlauf das Skript \texttt{./enter} mit dessen Hilfe man in die \textit{chroot} Umgebung wechseln kann. \\
Das Herzstück eines Arch Source Pakets ist die Datei \texttt{PKGBUILD}, in der alle Einstellungen zum Erstellen des Pakets enthalten sind. Details zum Aufbau finden sich unter \texttt{TODO: link suchen}. Alle benötigten Dateien werden zuerst in ein Verzeichnis innerhalb der \textit{Arch Linux} Umgebung kopiert. Nun wechselt man in die Umgebung (falls man sich nicht sowieso darin befindet), und wechselt dort in das Verzeichnis des Pakets. Dort wird nun \texttt{makepkg --asroot} ausgeführt. Wenn du ein Rechner mit nativem \textit{Arch Linux} hast, kannst du das Paket natürlich auch als normalen Benutzer erstellen. Dann ist die Option \texttt{--asroot} wegzulassen. \\
Bevor du dich daran machst ein eigenes Paket zu erstellen solltest du erst einmal schauen, ob dies nicht schon jemand vor dir gemacht hat und in das \textit{Arch User Repository} (\textit{AUR}) gestellt hat. Dafür gehst du zu \texttt{http://aur.archlinux.org} und suchst nach dem entsprechendem Paket. Binary Pakete werden im \textit{AUR} nicht angeboten. Deshalb gilt das angebotene \textit{TAR-Ball} herunterzuladen und in der \textit{Arch Linux} Umgebung zu entpacken und dann wie oben beschrieben zu kompilieren.

\end{document}
